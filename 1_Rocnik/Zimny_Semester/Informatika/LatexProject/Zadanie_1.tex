\documentclass{article}
\title{Uvod do informatiky\\
\bigbreak
\bigbreak
\bigbreak
\bigbreak
\bigbreak
\bigbreak
\bigbreak
\bigbreak
\bigbreak
\bigbreak
\bigbreak
\bigbreak
\bigbreak
\bigbreak
\bigbreak
\bigbreak
\bigbreak
\bigbreak
  Individualny projekt 1.
  \\\LaTeX{} \\
  Tema:
\bigbreak
\bigbreak
\bigbreak
\bigbreak
\bigbreak
\bigbreak
\bigbreak
\bigbreak
\bigbreak
\bigbreak
\bigbreak
\bigbreak
\bigbreak
\bigbreak
\bigbreak
\bigbreak
}
\author{Ing. Juraj Stefanovic  \\
	PEVS Fakulta informatiky  \\
	\and
	Jan Ondis \\
	PEVS Fakulta informatiky \\
	}

\date{\today
}
% Hint: \title{what ever}, \author{who care} and \date{when ever} could stand
% before or after the \begin{document} command
% BUT the \maketitle command MUST come AFTER the \begin{document} command!
\begin{document}

\maketitle

\section{Ako funguje kvantovy pocitac}\bigbreak
Klasicke pocitace, koduju informacie v binarnych ,,bitoch", ktore mozu mat hodnotu 0 s alebo 1 s. V kvantovom pocitaci je zakladnou jednotkou pamate kvantovy bit alebo qubit.
\bigbreak
Prvy napad kvantoveho pocitaca bolo pridanie tretieho stavu pomocou magnetu, kde magnet moze mat az tri stavy - pozitivny, negativny alebo ziaden. Neskor si ale ludia uvedomily, ze meranie udajov moze byt v akejkolvek forme. Dnes sa qubity sa vyrabaju pomocou fyzikalnych systemov, ako je spin elektronu alebo orientacia fotonu. Tieto systemy mozu byt v mnohych roznych usporiadaniach naraz, co je vlastnost znama ako kvantova superpozicia. Qubity mozu byt tiez neoddelitelne spojene pomocou javu nazyvaneho kvantove previazanie, kde 2 atomy vedia o sebe instantne, bez ohladu na ich poziciu vo vesmire. Takto vieme informacie prenasat a nie vysielat.
\bigbreak
Napriklad osem bitov staci na to, aby klasicky pocitac reprezentoval akekolvek cislo medzi 0 a 255. Ale osem qubitov staci na to, aby kvantovy pocitac reprezentoval kazde cislo medzi 0 a 255 sucasne. Niekolko stoviek zapletenych qubitov by stacilo na to, aby predstavovali viac cisel, ako je atomov vo vesmire.


\section{Introduction}
Make it possible for all to write documents with

\begin{abstract}
Short introduction to subject of the paper \ldots
\end{abstract}

\paragraph{Outline}
First we start with a little example of the article class, which is an
important documentclass. But there would be other documentclasses like
book \ref{book}, report \ref{report} and letter \ref{letter} which are
described in Section \ref{documentclasses}. Finally, Section
\ref{conclusions} gives the conclusions.



\section{Documentclasses} \label{documentclasses}

\begin{itemize}
\item article
\item book
\item report
\item letter
\end{itemize}


\begin{enumerate}
\item article
\item book
\item report
\item letter
\end{enumerate}

\begin{description}
\item[article\label{article}]{Article is \ldots}
\item[book\label{book}]{The book class \ldots}
\item[report\label{report}]{Report gives you \ldots}
\item[letter\label{letter}]{If you want to write a letter.}
\end{description}


\section{Conclusions}\label{conclusions}
There is no longer \LaTeX{} example which was written by \cite{doe}.


\begin{thebibliography}{9}
\bibitem[Doe]{doe} \emph{First and last \LaTeX{} example.},
John Doe 50 B.C.
\end{thebibliography}

\end{document}