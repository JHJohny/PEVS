\documentclass{article}
\title{Uvod do informatiky\\
\bigbreak
\bigbreak
\bigbreak
\bigbreak
\bigbreak
\bigbreak
\bigbreak
\bigbreak
\bigbreak
\bigbreak
\bigbreak
\bigbreak
\bigbreak
\bigbreak
\bigbreak
\bigbreak
\bigbreak
\bigbreak
  Individualny projekt 1.
  \\\LaTeX{} \\
  Kvantove pocitace
\bigbreak
\bigbreak
\bigbreak
\bigbreak
\bigbreak
\bigbreak
\bigbreak
\bigbreak
\bigbreak
\bigbreak
\bigbreak
\bigbreak
\bigbreak
\bigbreak
\bigbreak
\bigbreak
}
\author{Ing. Juraj Stefanovic  \\
	PEVS Fakulta informatiky  \\
	\and
	Jan Ondis \\
	PEVS Fakulta informatiky \\
	}

\date{\today
}
% Hint: \title{what ever}, \author{who care} and \date{when ever} could stand
% before or after the \begin{document} command
% BUT the \maketitle command MUST come AFTER the \begin{document} command!
\begin{document}

\maketitle

\section{Ako funguje kvantovy pocitac}
\bigbreak
Klasicke pocitace, koduju informacie v binarnych ,,bitoch", ktore mozu mat hodnotu 0 s alebo 1 s. V kvantovom pocitaci je zakladnou jednotkou pamate kvantovy bit alebo qubit.
\bigbreak
Prvy napad kvantoveho pocitaca bolo pridanie tretieho stavu pomocou magnetu.
Kde magnet moze mat az tri stavy - pozitivny, negativny alebo ziaden. Neskor si ale ludia uvedomily, ze meranie udajov moze byt v akejkolvek forme. Dnes sa qubity sa vyrabaju pomocou fyzikalnych systemov, ako je spin elektronu alebo orientacia fotonu. Tieto systemy mozu byt v mnohych roznych usporiadaniach naraz, co je vlastnost znama ako kvantova superpozicia. Qubity mozu byt tiez neoddelitelne spojene pomocou javu nazyvaneho kvantove previazanie, kde 2 atomy vedia o sebe instantne, bez ohladu na ich poziciu vo vesmire. Takto vieme informacie prenasat a nie vysielat.
\bigbreak
Napriklad osem bitov staci na to, aby klasicky pocitac reprezentoval akekolvek cislo medzi 0 a 255. Ale osem qubitov staci na to, aby kvantovy pocitac reprezentoval kazde cislo medzi 0 a 255 sucasne. Niekolko stoviek zapletenych qubitov by stacilo na to, aby predstavovali viac cisel, ako je atomov vo vesmire.

\section{Kvantovy pocitac vs bezny pocitac}
Kvantove pocitace spracovavaju informacie odlisne. Klasicke pocitace pouzivaju tranzistory, ktore su bud 1 alebo 0. Kvantove pocitace pouzivaju qubity, ktore mozu byt sucasne 1 alebo 0. Pocet prepojenych qubitov exponencialne zvysuje kvantovy vypoctovy vykon. Medzitym prepojenie viacerych tranzistorov zvysuje vykon iba linearne.
\bigbreak
Klasicke pocitace su najvhodnejsie na kazdodenne ulohy, ktore je potrebne vykonat pomocou pocitaca. Medzitym su kvantove pocitace skvele na vykonavanie simulacii a analyz udajov, napriklad na chemicke alebo liekove testy. Tieto pocitace vsak musia byt velmi dobre chladene - omnoho viac ako bezne pocitace. Su tiez ovela drahsie a narocnejsie na stavbu.
\bigbreak
Klasicke vypoctove pokroky zahrnaju pridavanie pamate na zrychlenie pocitacov. Medzitym kvantove pocitace pomahaju riesit komplikovanejsie problemy. Aj ked kvantove pocitace nemusia bezat v programe Microsoft Word lepsie alebo rychlejsie, mozu komplexnejsie problemy vykonavat rychlejsie.
\bigbreak

\begin{itemize}
\item Sundar Pichai, CEO Microsoftu: \enquote{'Now you can capture the complexity because you can keep track of many more states.'}
\end{itemize}

\section{Prakticke vyuzitie kvantoveho pocitacu}
Firma Google mina miliardy dolarov na svoj plan postavit svoj kvantovy pocitac do roku 2029. Spolocnost otvorila v Kalifornii kampus s nazvom Google AI, ktory jej ma pomoct splnit jej ciel. Google do tejto technologie investuje uz roky. Rovnako tak aj ine spolocnosti, ako napriklad Honeywell International (HON) a International Business Machine (IBM). IBM ocakava, ze v nasledujucich rokoch dosiahne velke milniky v oblasti kvantovej vypoctovej techniky.
\bigbreak
Zatial co niektore spolocnosti postavili osobne (hoci drahe) kvantove pocitace, na komercnej strane stale nie je nic dostupne. A je tu zaujem o kvantovu vypoctovu techniku   a jej technologiu, pricom JPMorgan Chase a Visa tuto technologiu skumaju. Po vyvinuti by Google mohol spustit kvantovu vypoctovu sluzbu prostrednictvom cloudu.
\bigbreak
Spolocnosti mozu tiez ziskat pristup ku kvantovej technologii bez toho, aby museli postavit kvantovy pocitac. IBM planuje do roku 2023 mat k dispozicii kvantovy pocitac s 1 000 qubitmi. IBM zatial umoznuje pristup k svojim pocitacom, ak su sucastou jeho kvantovej siete. Medzi tie, ktore su sucastou siete, patria vyskumne organizacie, univerzity a laboratoria.
\bigbreak
Microsoft tiez ponuka spolocnostiam pristup ku kvantovej technologii prostrednictvom platformy Azure Quantum. To je na rozdiel od Google, ktory nepredava pristup k svojim kvantovym pocitacom.
\bigbreak
Najpravdepodobnejsie vyuzitie kvantovych pocitacov:
\begin{itemize}
\item Kyberneticka bezpecnost
\item Vyvoj liekov
\item Financie
\item Lepsie baterie
\item Optimalizacia dopravy
\item Predpoved pocasia a zmena klimy
\item Umela inteligencia
\item Objavovanie novych prvkov
\item Lieky
\item Vakciny
\end{itemize}
\section{Introduction}




\section{Introduction}
Make it possible for all to write documents with

\begin{abstract}
Short introduction to subject of the paper \ldots
\end{abstract}

\paragraph{Outline}
First we start with a little example of the article class, which is an
important documentclass. But there would be other documentclasses like
book \ref{book}, report \ref{report} and letter \ref{letter} which are
described in Section \ref{documentclasses}. Finally, Section
\ref{conclusions} gives the conclusions.



\section{Documentclasses} \label{documentclasses}

\begin{itemize}
\item article
\item book
\item report
\item letter
\end{itemize}


\begin{enumerate}
\item article
\item book
\item report
\item letter
\end{enumerate}

\begin{description}
\item[article\label{article}]{Article is \ldots}
\item[book\label{book}]{The book class \ldots}
\item[report\label{report}]{Report gives you \ldots}
\item[letter\label{letter}]{If you want to write a letter.}
\end{description}


\section{Conclusions}\label{conclusions}
There is no longer \LaTeX{} example which was written by \cite{doe}.


\begin{thebibliography}{9}
\bibitem[Doe]{doe} \emph{First and last \LaTeX{} example.},
John Doe 50 B.C.
\end{thebibliography}

\end{document}