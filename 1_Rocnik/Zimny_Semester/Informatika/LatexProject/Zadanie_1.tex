\documentclass{article}
\usepackage{hyperref}
\usepackage{graphicx}
\title{Uvod do informatiky\\
\bigbreak
\bigbreak
\bigbreak
\bigbreak
\bigbreak
\bigbreak
\bigbreak
\bigbreak
\bigbreak
\bigbreak
\bigbreak
\bigbreak
\bigbreak
\bigbreak
\bigbreak
\bigbreak
\bigbreak
\bigbreak
  Individualny projekt 1.
  \\\LaTeX{} \\
  Kvantove pocitace
\bigbreak
\bigbreak
\bigbreak
\bigbreak
\bigbreak
\bigbreak
\bigbreak
\bigbreak
\bigbreak
\bigbreak
\bigbreak
\bigbreak
\bigbreak
\bigbreak
\bigbreak
\bigbreak
}
\author{Ing. Juraj Stefanovic  \\
	PEVS Fakulta informatiky  \\
	\and
	Jan Ondis \\
	PEVS Fakulta informatiky \\
	}

\date{\today
}
% Hint: \title{what ever}, \author{who care} and \date{when ever} could stand
% before or after the \begin{document} command
% BUT the \maketitle command MUST come AFTER the \begin{document} command!
\begin{document}

\maketitle

\section{Ako funguje kvantovy pocitac}
\bigbreak
Klasicke pocitace, koduju informacie v binarnych ,,bitoch", ktore mozu mat hodnotu 0 s alebo 1 s. V kvantovom pocitaci je zakladnou jednotkou pamate kvantovy bit alebo qubit.
\bigbreak
Prvy napad kvantoveho pocitaca bolo pridanie tretieho stavu pomocou magnetu.
Kde magnet moze mat az tri stavy - pozitivny, negativny alebo ziaden. Neskor si ale ludia uvedomily, ze meranie udajov moze byt v akejkolvek forme. Dnes sa qubity sa vyrabaju pomocou fyzikalnych systemov, ako je spin elektronu alebo orientacia fotonu. Tieto systemy mozu byt v mnohych roznych usporiadaniach naraz, co je vlastnost znama ako kvantova superpozicia. Qubity mozu byt tiez neoddelitelne spojene pomocou javu nazyvaneho kvantove previazanie, kde 2 atomy vedia o sebe instantne, bez ohladu na ich poziciu vo vesmire. Takto vieme informacie prenasat a nie vysielat.
\bigbreak
Napriklad osem bitov staci na to, aby klasicky pocitac reprezentoval akekolvek cislo medzi 0 a 255. Ale osem qubitov staci na to, aby kvantovy pocitac reprezentoval kazde cislo medzi 0 a 255 sucasne. Niekolko stoviek zapletenych qubitov by stacilo na to, aby predstavovali viac cisel, ako je atomov vo vesmire.

\section{Kvantovy pocitac vs bezny pocitac}
Kvantove pocitace spracovavaju informacie odlisne. Klasicke pocitace pouzivaju tranzistory, ktore su bud 1 alebo 0. Kvantove pocitace pouzivaju qubity, ktore mozu byt sucasne 1 alebo 0. Pocet prepojenych qubitov exponencialne zvysuje kvantovy vypoctovy vykon. Medzitym prepojenie viacerych tranzistorov zvysuje vykon iba linearne.
\bigbreak
Klasicke pocitace su najvhodnejsie na kazdodenne ulohy, ktore je potrebne vykonat pomocou pocitaca. Medzitym su kvantove pocitace skvele na vykonavanie simulacii a analyz udajov, napriklad na chemicke alebo liekove testy. Tieto pocitace vsak musia byt velmi dobre chladene - omnoho viac ako bezne pocitace. Su tiez ovela drahsie a narocnejsie na stavbu.
\bigbreak
Klasicke vypoctove pokroky zahrnaju pridavanie pamate na zrychlenie pocitacov. Medzitym kvantove pocitace pomahaju riesit komplikovanejsie problemy. Aj ked kvantove pocitace nemusia bezat v programe Microsoft Word lepsie alebo rychlejsie, mozu komplexnejsie problemy vykonavat rychlejsie.
\bigbreak

\begin{itemize}
\item Sundar Pichai, CEO Microsoftu: \enquote{'Quantum computing and AI can solve the biggest problems.'}
\end{itemize}
\section{Prakticke vyuzitie kvantoveho pocitacu}
Firma Google mina miliardy dolarov na svoj plan postavit svoj kvantovy pocitac do roku 2029. Spolocnost otvorila v Kalifornii kampus s nazvom Google AI, ktory jej ma pomoct splnit jej ciel. Google do tejto technologie investuje uz roky. Rovnako tak aj ine spolocnosti, ako napriklad Honeywell International (HON) a International Business Machine (IBM). IBM ocakava, ze v nasledujucich rokoch dosiahne velke milniky v oblasti kvantovej vypoctovej techniky.
\bigbreak
Zatial co niektore spolocnosti postavili osobne (hoci drahe) kvantove pocitace, na komercnej strane stale nie je nic dostupne. A je tu zaujem o kvantovu vypoctovu techniku   a jej technologiu, pricom JPMorgan Chase a Visa tuto technologiu skumaju. Po vyvinuti by Google mohol spustit kvantovu vypoctovu sluzbu prostrednictvom cloudu.
\bigbreak
Spolocnosti mozu tiez ziskat pristup ku kvantovej technologii bez toho, aby museli postavit kvantovy pocitac. IBM planuje do roku 2023 mat k dispozicii kvantovy pocitac s 1 000 qubitmi. IBM zatial umoznuje pristup k svojim pocitacom, ak su sucastou jeho kvantovej siete. Medzi tie, ktore su sucastou siete, patria vyskumne organizacie, univerzity a laboratoria.
\bigbreak
Microsoft tiez ponuka spolocnostiam pristup ku kvantovej technologii prostrednictvom platformy Azure Quantum. To je na rozdiel od Google, ktory nepredava pristup k svojim kvantovym pocitacom.

\bigbreak
Najpravdepodobnejsie vyuzitie kvantovych pocitacov:
\begin{itemize}
\item Kyberneticka bezpecnost
\item Vyvoj liekov
\item Financie
\item Vyvoj baterii
\item Optimalizacia dopravy
\item Predpoved pocasia a zmena klimy
\item Umela inteligencia
\item Objavovanie novych chemickych prvkov
\item Vakciny
\item Optimalizacia dodavatelskeho retazca
\end{itemize}

\section{Kvantova nadradenost (quantum supremacy)}
V roku 2012 profesor John Preskill vymyslel termin ,,kvantova nadradenost", aby opisal bod, ked sa kvantove pocitace stanu dostatocne vykonnymi na to, aby vykonali nejaku vypoctovu ulohu, ktoru klasicke pocitace nedokazali urobit v rozumnom casovom ramci. Zamerne nevyzadoval, aby vypoctova uloha bola uzitocna. Kvantova nadvlada je prechodnym milnikom, o ktory sa treba usilovat dlho predtym, ako bude mozne postavit velke kvantove pocitace na vseobecne ucely.
\bigbreak
Google tvrdi, ze preukazal nieco, co sa nazyva ,,kvantova nadradenost", v clanku publikovanom \href{https://www.nature.com/articles/s41586-019-1666-5}{v Nature}. To by znamenalo vyznamny milnik vo vyvoji noveho typu pocitaca, znameho ako kvantovy pocitac, ktory by dokazal vykonavat velmi narocne vypocty ovela rychlejsie ako cokolvek ine na beznych ,,klasickych" pocitacoch. Ale tim z IBM zverejnil svoj vlastny dokument, v ktorom tvrdi, ze dokaze reprodukovat vysledky Google na existujucich superpocitacoch.
\bigbreak
Zatial co Google verzus IBM moze byt dobrym pribehom, tato nezhoda medzi dvoma najvacsimi technologickymi spolocnostami na svete skor odvadza pozornost od skutocneho vedeckeho a technologickeho pokroku, ktory stoji za pracou oboch timov. Napriek tomu, ako to moze zniet, ani prekrocenie milnika kvantovej nadvlady by neznamenalo, ze kvantove pocitace sa chystaju prevziat kontrolu. Na druhej strane uz len priblizenie sa k tomuto bodu ma vzrusujuce dosledky pre buducnost technologie.
\bigbreak
\bigbreak
\includegraphics[width=\textwidth,height=\textheight,keepaspectratio]{https://d1e00ek4ebabms.cloudfront.net/production/f3571799-42d3-40f5-a506-1412e37c6388.jpg}
\bigbreak

\section{Zaver - kedy kvantove pocitace sa stanu beznymi pocitacmi}
Pocitace boli kedysi povazovane za spickovu technologiu, pristupnu iba vedcom a vyskolenym odbornikom. V druhej polovici 70. rokov vsak doslo v historii vypoctovej techniky k seizmickemu posunu. Nebolo to len tym, ze stroje sa stali ovela mensimi a vykonnejsimi - aj ked, samozrejme, stali sa. Bol to posun v tom, kto bude pouzivat pocitace a kde, pretoze boli k dispozicii kazdemu na pouzitie vo vlastnom dome.
\bigbreak
Dnes je kvantova vypoctova technika v plienkach. Kvantove vypocty zahrnaju niektore z najzaujimavejsich konceptov fyziky 20. storocia. V USA, Google, IBM a NASA experimentuju a stavaju prve kvantove pocitace. Cina tiez intenzivne investuje do kvantovych technologii.
\bigbreak
Verim, ze dojde k analogickemu posunu ku kvantovym pocitacom, kde sa nadsenci budu moct hrat s kvantovymi pocitacmi zo svojich domovov. A tento posun nastane ovela skor, ako si vacsina ludi uvedomuje.
\bigbreak
Nemyslim si vsak, ze by malo zmysel spekulovat o tom, co bude vacsina ludi robit s kvantovymi pocitacmi o 50 rokov. Mozno by bolo zmysluplnejsie opytat sa, kedy sa kvantova vypoctova technika stane niecim, co moze ktokolvek pouzivat z vlastneho domova.
\bigbreak
Odpoved je, ze je to uz mozne. V roku 2016 spolocnost IBM pridala do cloudu maly kvantovy pocitac. Kazdy, kto ma pripojenie na internet, moze na tomto pocitaci navrhnut a spustit svoje vlastne kvantove obvody. Kvantovy obvod je sekvencia zakladnych krokov, ktore vykonavaju kvantovy vypocet.
\bigbreak
Kvantovy pocitac IBM je nielen volne pouzitelny, ale tento kvantovy pocitac ma aj jednoduche graficke rozhranie. Je to maly, nie velmi vykonny stroj - podobne ako prve domace pocitace - ale fanusikovia mozu zacat hrat. Posun sa zacal.
\bigbreak
Ludia vstupuju do veku, kedy je jednoduche ucit sa a experimentovat s kvantovymi vypoctami. Rovnako ako pri prvych domacich pocitacoch nemusi byt jasne, ze existuju problemy, ktore je potrebne vyriesit pomocou kvantovych pocitacov, ale ked sa ludia hraju, je pravdepodobne, ze zistia, ze potrebuju viac energie a funkcii. Tym sa otvori cesta pre nove aplikacie, ktore sme si doteraz nevedeli predstavit.

\end{document}